\documentclass[11pt,a4paper]{article}
\usepackage{amsmath,amssymb,geometry,booktabs}
\usepackage{graphicx}
\usepackage[style=authoryear,backend=bibtex]{biblatex}
\addbibresource{references.bib}
\graphicspath{{figures/}}
\usepackage{subfig}
\usepackage{hyperref}
\geometry{margin=1in}
\title{Recursive Cosmology and Cosmic Memory: A Fractal-Torsion Framework for Universe Evolution}
\author{Brett Harris \\ rellikss@outlook.com}
\date{}

\begin{document}

\maketitle

\section*{Abstract}

This paper introduces a novel cosmological model in which the universe evolves recursively through black hole bounces, carrying forward memory in the form of torsion and spin alignment. We formalize a Lagrangian framework combining quantum potential, torsion fields, and entropy surfaces, then simulate this geometry across generational epochs. The model explains cosmic acceleration, spin correlations, entropy growth, and early galactic maturity without invoking dark matter or dark energy. It provides a visual, testable structure for a multiverse nested in spin-encoded causality.

\section{Introduction}

Recent astrophysical data \cite{Shamir2020,McCaffrey2025,Schwarz2016} — including early galaxy formation, spin alignments across large scales, and entropy bounds — suggest the standard $\Lambda$CDM model may be incomplete. This paper proposes a recursive, memory-preserving cosmology built from first principles of field geometry and holography \cite{Poplawski2010,Poplawski2012}.

We construct a total Lagrangian integrating entropy surface terms, torsion-based spin memory, and fractal quantum potentials \cite{Holland1993,Bohm1952}. These are simulated visually to map how structure and information evolve through a bounce cycle — from collapse to rebirth.

\section{Lagrangian Framework}

We define the total Lagrangian:

\[ 
\mathcal{L}_{\text{total}} = \frac{1}{\rho^2} \left[ \alpha A \rho^2 \delta(x,y) + \beta \left( \frac{1}{4} \rho \nabla^2 \rho - \frac{1}{8} |\nabla \rho|^2 \right) + \frac{\lambda}{4} \rho |\nabla \rho|^2 \right] 
\]

where: $\delta(x, y)$ encodes entropy at holographic surfaces, $\nabla^2 \rho$ describes the fractal quantum potential $Q$ \cite{Holland1993}, $|\nabla \rho|^2$ captures torsion-based spin memory \cite{Poplawski2010}, and $\alpha, \beta, \lambda$ are tunable geometric coefficients.

The resulting Euler-Lagrange equation propagates density and curvature through bounces \cite{Alam2025,Tukhashvili2023}.

\section{Simulations}

Eight simulations illustrate this model:
\begin{enumerate}
    \item 1D Bounce Mechanics: collapse and quantum rebound
    \item 2D Field Geometry: curvature and structure emergence
    \item Recursive Lineage Tree: universe birth from black holes
    \item Fractal Potential Dynamics: structural resonance patterns
    \item Field Propagation: Lagrangian-driven structure evolution
    \item Entropy Shock Response: memory dynamics from causal events
    \item Spin Alignments: large-scale torsion coherence
    \item Holographic Shell Growth: entropy-bound horizon expansion
\end{enumerate}

\begin{figure}[htbp]
\centering
\subfloat[Sim 1: 1D Bounce Mechanics]{{\includegraphics[width=0.45\textwidth]{sim1_bounce_data.png}}}
\subfloat[Sim 2: 2D Field Geometry]{{\includegraphics[width=0.45\textwidth]{sim2_geometry_data.png}}}\\
\subfloat[Sim 3: Recursive Lineage Tree]{{\includegraphics[width=0.45\textwidth]{sim3_lineage_data.png}}}
\subfloat[Sim 4: Fractal Potential Dynamics]{{\includegraphics[width=0.45\textwidth]{sim4_fractal_data.png}}}\\
\subfloat[Sim 5: Field Propagation]{{\includegraphics[width=0.45\textwidth]{sim5_propagation_data.png}}}
\subfloat[Sim 6: Entropy Shock Response]{{\includegraphics[width=0.45\textwidth]{sim6_shock_data.png}}}\\
\subfloat[Sim 7: Spin Alignments]{{\includegraphics[width=0.45\textwidth]{sim7_spin_data.png}}}
\subfloat[Sim 8: Holographic Shell Growth]{{\includegraphics[width=0.45\textwidth]{sim8_shell_data.png}}}
\caption{Simulations illustrating the model. All figures generated from the Lagrangian framework with fixed parameters $\alpha=1.2$, $\beta=0.6$, $\lambda=0.23$.}
\label{fig:sims}
\end{figure}

\section{Cosmic Memory and Entropy}

We extend the model to simulate spin inheritance across generational universes \cite{Tukhashvili2023}. Results show:
\begin{itemize}
    \item Memory fields degrade slowly but persist
    \item Entropy accumulates asymmetrically, directing the arrow of time
    \item Geometry functions as both dynamic and hereditary code
\end{itemize}

\section{Implications}

This model:
\begin{itemize}
    \item Provides an alternative to inflation and dark energy \cite{Shen2024,Tsagas2025}
    \item Matches observations of early galaxy formation \cite{McCaffrey2025}
    \item Predicts testable spin correlations \cite{Tempel2013,Shamir2020}
    \item Embeds time and structure in geometry
    \item Aligns with holographic and causal boundary theories \cite{Swingle2012,VanRaamsdonk2010}
\end{itemize}

\section{AdS/CFT Embedding and Emergent Time}

We map the radial geometry of our recursive cosmology into an AdS-like shell space \cite{Rickles2011,Takayanagi2025}. Entropy resides on boundary shells, consistent with the Bekenstein-Hawking area law, while torsion aligns with periodic angular modes, behaving like conformal field operators.

By decomposing boundary entropy into angular Fourier modes, we observe discrete peaks representing quantized complexity \cite{Swingle2012}. These increasing entropy modes mirror the layering of causal sets and tensor networks in quantum gravity — suggesting that time itself emerges from entanglement complexity \cite{VanRaamsdonk2010}.

To formalize this: Let $S(\theta, r)$ represent the entropy shell function. Then the emergent time parameter $\tau$ may be defined geometrically as:

\[
\tau \sim \int_{r_0}^{r} C[S(\theta, r')] \, dr'
\]

where $C$ is a complexity or mode-counting function over angular entropy.

This entropy-angular decomposition implies an AdS/CFT-type duality:
\begin{itemize}
    \item Bulk geometry encodes structure (via spin and $Q$)
    \item Boundary entropy shells act as dynamic memory screens
    \item Time emerges as a radial foliation of increasing complexity
\end{itemize}

\section{Observable Consequences and Predictions}

We test the model’s output against observational signatures across three cosmological scales:

\begin{itemize}
    \item \textbf{Spin Coherence}: Torsion-inherited geometry produces galaxy spin alignments across large spatial scales \cite{McAdam2023,Shamir2020}
    \item \textbf{CMB Anisotropies}: Boundary entropy fluctuations map to low-$\ell$ multipole mode deviations \cite{Patel2025,Schwarz2016}
    \item \textbf{Primordial Structure Seeding}: Entropy gradients post-bounce seed early galaxy formation \cite{McCaffrey2025,Tsagas2025}
\end{itemize}

\section{Dark Matter Analogue from Geometry}

We simulate three major observables using only torsion, entropy, and fractal potential:
\begin{itemize}
    \item \textbf{Galaxy Rotation Curves}: Match flat velocity profiles \cite{Poplawski2012}
    \item \textbf{Gravitational Lensing Maps}: Replicate lensing via geometric $Q$ \cite{Alam2025}
    \item \textbf{Entropy-Induced Halos}: Coherent structures emerge from entropy geometry \cite{Kadam2025}
\end{itemize}

\section{$\Lambda$CDM Comparison and Evaluation}

\begin{table}[h]
\centering
\begin{tabular}{lcc}
\toprule
Observable & $\Lambda$CDM Prediction & Fractal-Torsion Model \\
\midrule
Galaxy Rotation & Flat due to CDM halo & Flat due to torsion \\
Gravitational Lensing & Strong via invisible mass & Strong via $Q$ \\
Halo Distribution & CDM filament clustering & Entropy shell memory \\
Early Galaxy Formation & Requires tuning & Emerges naturally \\
CMB Low-$\ell$ Modes & Weak dipole & Strong from entropy waves \\
\bottomrule
\end{tabular}
\caption{Comparison of predictions with standard and proposed models.}
\end{table}

\section{Summary and Outlook}

We propose a new cosmological engine based on recursive structure, memory propagation, and entropy encoding \cite{Poplawski2010,Shamir2020}. The model:
\begin{itemize}
    \item Explains cosmic acceleration, spin alignment, and early galaxies without dark matter or dark energy
    \item Derives emergent time from entropy mode complexity
    \item Matches $\Lambda$CDM outputs using only geometric and field-theoretic mechanisms
\end{itemize}

If validated, this model could:
\begin{itemize}
    \item Shift how we view cosmological constants
    \item Recast black holes as cosmogenic transitions
    \item Redefine time, mass, and spin as emergent from causal layerings
\end{itemize}

\section{Future Work}

\begin{itemize}
    \item Formalize the connection to AdS/CFT and scale relativity
    \item Quantify entropy/structure inheritance metrics \cite{Tukhashvili2023}
    \item Explore observational footprints in CMB and galaxy rotation
    \item Simulate lightcone propagation from bounce and compare to JWST/Planck data
    \item Expand torsion potential into topological classifications
    \item Animate bounce-shell simulations for outreach and publications
    \item Prepare GitHub/data repository for reproducibility and open science
\end{itemize}

\printbibliography

\end{document}