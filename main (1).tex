
\documentclass[12pt]{article}
\usepackage{amsmath, amssymb, graphicx, hyperref, geometry}
\geometry{margin=1in}
\title{Recursive Cosmology and Cosmic Memory:\\
A Fractal-Torsion Framework for Universe Evolution}
\author{Brett Harris \\
\texttt{rellikss@outlook.com}}
\date{}

\begin{document}

\maketitle

\begin{abstract}
This paper introduces a novel cosmological model in which the universe evolves recursively through black hole bounces, carrying forward memory in the form of torsion and spin alignment. We formalize a Lagrangian framework combining quantum potential, torsion fields, and entropy surfaces, then simulate this geometry across generational epochs. The model explains cosmic acceleration, spin correlations, entropy growth, and early galactic maturity without invoking dark matter or dark energy.
\end{abstract}

\tableofcontents
\newpage

\section{Introduction}
Recent astrophysical data---including early galaxy formation, spin alignments across large scales, and entropy bounds---suggest the standard $\Lambda$CDM model may be incomplete. This paper proposes a recursive, memory-preserving cosmology built from first principles of field geometry and holography.

\section{Lagrangian Framework}
We define the total Lagrangian as follows:

\begin{equation}
\mathcal{L}_{\text{Total}} = \frac{1}{\rho^2} \left[\alpha A \rho^2 \delta(x, y) + \beta \left(\frac{1}{4} \rho \nabla^2 \rho - \frac{1}{8} |\nabla \rho|^2 \right) + \frac{\lambda}{4} \rho |\nabla \rho|^2\right]
\end{equation}

\begin{itemize}
  \item $\delta(x, y)$ encodes entropy concentration at causal surfaces
  \item $\nabla^2 \rho$ models the fractal quantum potential $\mathcal{Q}$
  \item $|\nabla \rho|^2$ represents torsion and spin alignment memory
  \item $\alpha, \beta, \lambda$ are field coefficients governing balance
\end{itemize}

We derive the Euler-Lagrange equation accordingly, embedding quantum surface dynamics and spinor complexity into an effective geometric potential.

\section{Simulation Outputs}
\begin{itemize}
  \item \textbf{Rotation curves}: reproduced from torsion field geometry
  \item \textbf{Gravitational lensing}: derived from quantum potential $\mathcal{Q}$
  \item \textbf{Entropy halo formation}: from nonlinear bounce collapse
\end{itemize}

\begin{figure}[h!]
\centering
\includegraphics[width=0.75\textwidth]{figures/rotation_curve_torsion.png}
\caption{Galaxy rotation curve reproduced from torsion field, matching dark matter profiles.}
\end{figure}

\begin{figure}[h!]
\centering
\includegraphics[width=0.75\textwidth]{figures/lensing_Q.png}
\caption{Gravitational lensing from inherited quantum potential field.}
\end{figure}

\begin{figure}[h!]
\centering
\includegraphics[width=0.75\textwidth]{figures/halo_entropy_shell.png}
\caption{Entropy-driven halo formation post-bounce, mimicking CDM clustering.}
\end{figure}

\section{Quantum Layer Extensions}
We simulate three quantum-correspondence mechanisms:

\begin{itemize}
  \item \textbf{Spin Torsion Fields} as SU(2) vector connections
  \item \textbf{Quantum Bounce} as a superposed process matrix
  \item \textbf{Shell Entropy} from von Neumann entanglement measures
\end{itemize}

These show that complexity and structure propagation may arise directly from angular entanglement layering---suggesting a quantum origin for spacetime and emergent time itself.

\section{Discussion}
The implications of this model reach into emergent time, entanglement, black hole cosmogenesis, and dark matter geometry. It provides testable alternatives to inflation, $\Lambda$CDM, and conventional Big Bang models.

\section*{Contact}
Brett Harris — \texttt{rellikss@outlook.com}

\end{document}
